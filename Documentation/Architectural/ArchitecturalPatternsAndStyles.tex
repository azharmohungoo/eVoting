The project implementaton will use MVC (Model View controller) pattern for the project. As the following benefits will be utilised provided by implementing MVC
	\begin{enumerate}
		\item Separation of concerns thus allowing:
		\begin{enumerate}
			\item Re-use of the business logic across the platform 
			\item Multiple user interfaces can be developed without concerning the codebase
		\end{enumerate}
		
		\item Developer specialisation and focus:
		\begin{enumerate}
			\item The developers of UI can focus exclusively on the UI screens without bogged down with business logic.
			\item The developer of Model / business can focus exclusively on the business logic implementations, modifications, updations without concerning the look and feel and it has nothing to with business logic.
		\end{enumerate}
		
		\item Parallel development by separate teams:
		\begin{enumerate}
			\item Business logic developers can build the classes, while the UI developers can involve in designing UI screens simultaneously, resulting the interdependency issues and time conservation.
			\item UI updations can be made without slowing down the business logic process.
			\item Business logic rules changes are very less that needs the revision / updations of the UI.
		\end{enumerate}
		
		\item Multiple view support:
		\begin{enumerate}
			\item Due to the separation of the model from the view, the user interface can display multiple views of the same data at the same time.
		\end{enumerate}
		
		\item Change Accommodation:
		\begin{enumerate}
			\item User interfaces tend to change more frequently than business rules. (different colors, fonts, screen layouts, and levels of support for new 									devices such as cell phones or PDAs) Because the model does not depend on the views, adding new types of views to the system generally does not affect the model. As a result, the scope of change is confined to the view.
		\end{enumerate}
	\end{enumerate}