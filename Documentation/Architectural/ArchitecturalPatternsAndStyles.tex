Electronic Voting will make use of a combination between the Client-Server and Component-Based architecture styles. The system will be divided into a 'client side' and a 'server side', with a RESTful web service running on the server side as a communication medium between the two. Component-Based architecture style will be used on the server side, where each component contains everything necessary to execute only one desired aspect of the system.
We will be using Component-Based architecture to separate all of the functionality for Voter, Activator, Party, Admin, Security, Blockchain and Database into their own respective modules. Reasons for Client-Server style:
	\begin{enumerate}
		\item Accessibility:
		\begin{enumerate}
			\item eVoting can be accessed form various platforms on a local network or over the Internet.
			\item Our server can be accessed remotely by administrators or support teams.
		\end{enumerate}
		\item Adaptability and Scalability:
		\begin{enumerate}
			\item Performance changes on the server can be easily made by upgrading the server.
			\item New resources can be added to the server.
		\end{enumerate}
		\item Pluggability:
		\begin{enumerate}
			\item We are using a RESTful web service on the server end.
			\item Because of this, it allows us add any number of client platform types (eg iOS, Windows apps, or even other web applications) to integrate with our system if they support our communication medium.
		\end{enumerate}
	\end{enumerate}
	
Reasons for using a Component-Based architecture style:
	\begin{enumerate}
		\item Separation of concern (modularization):
		\begin{enumerate}
			\item Re-use of the business logic across the platform.
			\item Multiple modules can be developed without affecting the other modules.
		\end{enumerate}
		
		\item Developer specialisation and focus:
		\begin{enumerate}
			\item A developer can focus exclusively on one part of the business logic while another developer can focus on other business logic.
			\item A developer that has more experience and better skill levels in a certain module, can give better input.
			\item If there is a bug in a module, only that module needs attention.
		\end{enumerate}
		
		\item Parallel development by separate teams:
		\begin{enumerate}
			\item Business logic developers can build the classes, while the UI developer(s) can involve in designing UI screens simultaneously.
			\item UI updates can be made without slowing down the business logic process.
		\end{enumerate}
		
		\item Multiple view support:
		\begin{enumerate}
			\item Due to the separation of the model from the view, the user interface can display multiple views of the same data at the same time.
		\end{enumerate}
		
		\item Testing:
		\begin{enumerate}
			\item Different business logic modules can be tested independently of other modules by using integration and unit testing.
			\item Each module can expect that the other modules have a good level of correctness, thus expecting a pre-defined output.
		\end{enumerate}
	\end{enumerate}
