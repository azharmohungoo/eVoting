\begin{enumerate}
	\item \textbf{Convenience:} The system shall allow the voters to cast their votes quickly, in one session, and should not require many special skills or intimidate the voter.
	
	\item \textbf{User-Interface:} The system shall provide an easy-to-use user-interface. Also, it shall not disadvantage any candidate while displaying the choices.
	
	\item \textbf{Transparency:} Voters should be able to possess a general knowledge and understanding of the voting process.
	
	\item \textbf{Accuracy:} The system shall record and count all the votes and shall do so correctly.
	
	\item \textbf{Eligibility:} Only authorized voters, who are activated, should be able to vote.
	
	\item \textbf{Uniqueness:} No voter should be able to vote more than once for the same poll.
	
	\item \textbf{Auditability:} It should be possible to verify that all votes have been correctly accounted for in the final election tally, and there should be reliable and demonstrably authentic election records, in terms of physical, permanent audit trail, which should not reveal the user’s identity in any manner.
	
	\item \textbf{Confirmation:} The voter shall be able to confirm clearly how his vote is being cast, and shall be given a chance to modify his vote before he commits it.
	
	\item \textbf{No Over-voting:} The voter shall be prevented from choosing more than one party.
	
	\item \textbf{Documentation and Assurance:} The design, implementation, and testing procedures must be well documented so that the voter-confidence in the election process is ensured.
	
	\item \textbf{Cost-effectiveness:} Election systems should be affordable and efficient.
	
	\item \textbf{Authenticity:} Ensure that the voter must identify himself (with respect to the registration database) to be entitled to vote.
	
	\item \textbf{Anonymity:} Ensure that votes must not be associated with voter identity.
	
	\item \textbf{System Integrity:} Ensure that the system cannot be re-configured during operation.
	
	\item \textbf{Data Integrity:} Ensure that each vote is recorded as intended and cannot be tampered with in any manner, once recorded. Votes should not be modified, forged or deleted without detection.
	
	\item \textbf{Privacy:} No one should be able to determine how any individual voted.
	
	\item \textbf{Reliability:} Election systems should work robustly, without loss of any votes, even in the face of numerous failures, including failures of voting machines and total loss of network communication. The system shall be developed in a manner that ensures there is no malicious code or bugs.
	
	\item \textbf{Availability:} Ensure that system is protected against accidental and malicious denial of service attacks.
	
	\item \textbf{Simplicity:} The system shall be designed to be extremely simple, as complexity is the enemy of security.
	
	\item \textbf{System Accountability:} Ensure that system operations are logged and audited.
	
	\item \textbf{Authentication and Control:} Ensure that those operating and administering the system are authenticated and have strictly controlled functional access on the system.
	
	\item \textbf{Distribution of Authority:} The administrative authority shall not rest with a single entity. The authority shall be distributed among multiple administrators, who are known not to collude among themselves.
\end{enumerate}
